\documentclass{oci}
\usepackage[utf8]{inputenc}
\usepackage{lipsum}

\title{Reacciones}

\newcommand{\platform}{Harmony}
\newcommand{\hero}{Jorge}

\begin{document}
\begin{problemDescription}
\platform{} es una conocida plataforma de mensajería instantánea por texto en la que los usuarios pueden comunicarse entre pares o grupos.
Entre sus muchas características, es posible reaccionar con emojis a los mensajes de otros miembros de las conversaciones.
Cada miembro puede reaccionar con uno o más emojis a cualquier mensaje, pero solo una vez con cada emoji.

Las reacciones aparecen debajo de los mensajes, de forma horizontal, dispuestos de izquierda a derecha y junto a un contador.
Si un emoji ya ha sido usado antes, entonces el contador aumenta; si no ha sido antes, se agrega al final con un contador en 1.

\hero{} es un usuario frecuente de \platform{}, y suele recurrir a las reacciones para responder a los mensajes.
Esto es posible en algunos casos ya que \platform{} tiene emojis por cada letra del abecedario en $k$ colores diferentes.

A \hero{} le gustaría determinar si es posible o no escribir un mensaje con estas reacciones antes de empezar a intentarlo, de lo contrario, podría ocurrir que empiece a formar el mensaje con las reacciones y quede incompleto o a medias, y su reputación en \platform{} se vería afectada.

\end{problemDescription}

\begin{inputDescription}
La entrada consiste en dos líneas.

La primera línea contiene dos enteros $n$ y $k$ ($1 \le k \le n \le 1\,000\,000$) que corresponden respectivamente al largo del mensaje que \hero{} quiere escribir con emojis y al número de colores (o variantes) que tiene cada letra del abecedario.

La segunda línea contiene $n$ letras del abecedario inglés (es decir, sin la \textbf{ñ}), en minúsculas y sin espacios entre sí.
\end{inputDescription}

\begin{outputDescription}
La salida consiste en una sola línea.

Si existe una forma de formar el mensaje, la línea debe contener $n$ enteros $c_i$ ($1 \le c_i \le k$) separados por espacios, donde el $i$-ésimo entero indica el color del que debe ser pintada la $i$-ésima letra del mensaje.

Si no es posible escribir el mensaje, la línea debe contener la palabra \texttt{imposible}.
\end{outputDescription}

\begin{scoreDescription}
  \subtask{20}
  Se probarán varios casos en que $k = 1$.
  \subtask{30}
  Se probarán varios casos en que $k = 2$.
  \subtask{50}
  Se probarán varios casos sin restricciones adicionales.
\end{scoreDescription}

\begin{sampleDescription}
\sampleIO{sample-1}
\sampleIO{sample-2}
\end{sampleDescription}

\end{document}
