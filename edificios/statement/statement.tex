\documentclass{oci}
\usepackage[utf8]{inputenc}

\title{Parkour}

\begin{document}
\begin{problemDescription}
  Tras muchos años de esfuerzo y dedicación, Sebastián finalmente logró
  clasificar a la fase final de la Increíble Competencia de Parkour con
  Cohetes (ICPC).
  %
  En el último desafío, Sebastián debe arriesgar
  su vida en la ruta de las torres infernales.

  La ruta está compuesta de $N$ torres cada una con
  $M$ pisos, numerados de abajo hacia arriba entre 0
  y $M-1$.
  %
  Todos los participantes comienzan en el último piso de la primera
  torre (el piso $M-1$) y para completar la ruta deben saltar de torre
  en torre hasta alcanzar cualquier piso de la última torre.

  Para llegar de una torre a la siguiente, Sebastián puede saltar a un piso
  que esté a la misma o menor altura que el actual, o puede utilizar \emph{cohetes}
  para propulsarse hacia un piso más alto, utilizando un cohete por cada piso que asciende.
  %
  Por ejemplo, si Sebastián se encuentra en el piso 5 y desea llegar
  al piso 3 de la siguiente torre, puede hacerlo sin utilizar cohetes.
  %
  Pero si quiere saltar hasta el piso 8 necesitará utilizar 3 cohetes.

  Esta es una competencia muy extrema, por lo que muchos de los pisos
  estarán en llamas.
  %
  Como Sebastián desea salir vivo de esta, solo saltará a los pisos
  seguros, es decir, los pisos que no están en llamas.

  Además de ser un excelente deportista, Sebastián es una persona muy
  preocupada por el medio ambiente, por lo que desea reducir su huella
  de carbono minimizando la cantidad de cohetes que utiliza.
  %
  Para lograrlo, le gustaría saber cuál es la mínima cantidad
  de cohetes que necesita para completar la ruta.
  %
  ¿Podrás ayudarlo a cumplir su meta?

\end{problemDescription}

\begin{inputDescription}
  La primera línea de entrada contiene dos enteros $N$ y $M$ ($2 \leq N, M \leq 100$)
  correspondientes respectivamente a la cantidad de torres y la cantidad de pisos.

  A continuación vienen $N$ líneas.
  %
  La i-ésima línea describe el estado de los pisos de la i-ésima torre.
  %
  Cada línea comienza con un entero $P$ ($1 \leq P \leq M$) correspondiente
  a la cantidad de pisos seguros en la torre.
  %
  Luego siguen $P$ enteros $p$ ($0 \leq p < M$) correspondientes al
  número de cada piso accesible.

  Se garantiza que siempre es posible completar la ruta, y que el
  último piso de la primera torre es seguro.
\end{inputDescription}

\begin{outputDescription}
  La salida debe contener un único entero correspondiente a la cantidad mínima
  de cohetes necesarios para completar la ruta.
\end{outputDescription}

\begin{scoreDescription}
  \subtask{??}
  Se probarán varios casos en que ($2 \leq N, M \leq 5$).
  \subtask{??}
  Se probarán varios casos en que $M=2$.
  \subtask{??}
  Se probarán varios casos sin restricciones adicionales.
\end{scoreDescription}

\begin{sampleDescription}
\sampleIO{sample-1}
\sampleIO{sample-2}
\end{sampleDescription}

\end{document}
