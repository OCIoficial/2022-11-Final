\documentclass{oci}
\usepackage[utf8]{inputenc}
\usepackage{lipsum}

\title{Parkour}

\begin{document}
\begin{problemDescription}
  Tras muchos años de esfuerzo y dedicación, Sebastián finalmente logró llegar a la última etapa de la increíble competencia de parkour con cohetes (ICPC).

  En la prueba final, los participantes deberán saltar a través de una fila de n torres, pasando de una torre a la siguiente, procurando pasar por cada una de ellas, y al final de este, a los pocos que logren completar el camino serán trasladados al campamento, donde serán entrenados para participar en la competencia mundial.

  Cada torre contiene m pisos. Para llegar a la siguiente torre, Sebastián puede saltar a un piso que esté a la misma o menor altura que el actual, o puede utilizar cohetes para saltar a un piso más alto, para ello, necesita utilizar un cohete para cada piso que suba.

  Por ejemplo, si actualmente se encuentra en el piso 5 de la tercera torre y quiere llegar al piso 3 de la cuarta torre, puede hacerlo sin utilizar cohetes. Pero si quiere saltar hasta el piso 8 de la cuarta torre, necesitará utilizar 3 cohetes.

  El problema es que muchos pisos están en llamas. Por lo que para poder saltar de una torre a otra sin morir en el intento, deberá aterrizar en un piso seguro.

  Todos los participantes comienzan en el último piso de la primera torre, y deben llegar a la última.

  Además de ser un excelente deportista, Sebastián es una persona muy preocupada por el medio ambiente, por lo que desea reducir su huella de carbono minimizando la cantidad de cohetes que utiliza. Para esto, te ha pedido a ti que diseñes un programa que le permita encontrar la mínima cantidad de cohetes que necesita para llegar al final del camino.

Tiene 3 subtareas:
- Un n y h que te permitan hacer brureforce
-h=2, para que puedas hacer una solución greedy
-la solución completa, que es hacer un Dijkstra con el grafo implícito
\end{problemDescription}

\begin{inputDescription}
  La primera linea de entrada contiene dos enteros $N$ y $M$ ($2 \leq N, M \leq 10000$) correspondientes respectivamente a la cantidad de torres y la cantidad de pisos.

  A continuación, vienen $N$ líneas. La i-ésima línea describe el estado de los pisos de la i-ésima torre. Primero, contiene un entero $P$ ($1 \leq P \leq M$), la cantidad de pisos accesibles (es decir, que no están en llamas). Y luego le siguen P enteros p ($0 \leq p < M$), el índice de los pisos.
\end{inputDescription}

\begin{outputDescription}
  La salida debe contener un único entero, la cantidad mínima de cohetes necesaria para llegar hasta la última torre.
\end{outputDescription}

\begin{scoreDescription}
  \subtask{??j}
  Se probarán varios casos en que ($2 \leq N, M, 5$).
  \subtask{??}
  Se probarán varios casos en que $M=2$.
  \subtask{??}
  Se probarán varios casos sin restricciones adicionales.
\end{scoreDescription}

\begin{sampleDescription}
\sampleIO{sample-1}
\sampleIO{sample-2}
\end{sampleDescription}

\end{document}
